\documentclass[12pt]{article}
\usepackage{amssymb,amsmath,amsthm}
\usepackage{graphicx,tikz}
\usepackage{fancyhdr,enumerate}
\usepackage{faktor}
\usepackage{color,soul}
\usepackage{marginnote}

\setlength{\headheight}{28pt}
\usepackage[top=1.0in, bottom=0.5in, left=0.75in, right=2.5in]{geometry}
\pagestyle{fancy}
\fancyhf{}
\lhead{MATH3004 HW \#7\\ Fall 2017}
\rhead{\thepage}
\chead{Rob Ireton}

\title{Homework 7}

\newcommand{\zee}{\mathbb{Z}}
\newcommand{\Q}{\mathbb{Q}}
\newcommand{\arr}{\mathbb{R}}
\newcommand{\C}{\mathbb{C}}
\newcommand{\such}{\text{ s.t. }}
\newcommand{\WTS}[1]{ \textcolor[rgb]{0,0.33,0}{$\blacktriangleright$ We want to show:  #1 $\blacktriangleleft$}  }

\begin{document}



\begin{itemize}
\item[\textbf{ 1.3.33.}] Let $p>1$. If $2^p-1$ is prime, prove that $p$ is prime.
[Hint: Prove the contrapositive.]
\begin{proof}
	\dots
\end{proof}

\item[\textbf{3.1.35a.}] Prove or disprove: If $R$ and $S$ are integral domains, then $R\times S$ is an integral domain.
\begin{proof}
	\dots
\end{proof}

\item[\textbf{3.1.35b.}] Prove or disprove: If $R$ and $S$ are fields, then $R\times S$ is a field.
\begin{proof}
	\dots
\end{proof}

\item[\textbf{ 3.2.21a.}] Let $R$ be a ring and let $a\in R$ be a nonzero element that is not a zero divisor. Prove that $\forall b,c\in R$, \[ab=ac\implies b=c.\]
\begin{proof}
	\dots
\end{proof}

\item[\textbf{3.2.29.}] Let $R$ be a commutative ring with identity. Prove that $R$ is an integral domain $\iff$ cancellation holds in $R$.
[``cancellation holds in $R$'' means that $\forall  a,b,c\in R$, if $a\ne 0_R$ and $ab=ac$ then $b=c$.]
\begin{proof}
	\dots
\end{proof}

\item[\textbf{ 4.1.17.}] Let $R$ be an integral domain. Assume that the Division Algorithm always hols in $R[x]$. Prove that $R$ is a field.
\begin{proof}
	\dots
\end{proof}

\item[\textbf{ 4.4.24.}] Fix $a\in F$ and define \[\varphi_a : F[x]\to F \text{\;\; by \;\;} \varphi_a\bigl(f(x)\bigr) =f(a).\] Prove that $\varphi_a$ is a surjective ring homomorphism.
\begin{proof}
	\dots
\end{proof}

\item[\textbf{5.1.2.}] Let $p(x)$ be a nonzero constant polynomial in $F[x]$. Prove that any two polynomials in $F[x]$ are congruent modulo $p(x)$.
\begin{proof}
	\dots
\end{proof}

\item[\textbf{5.1.8.}] Prove or disprove: If $p(x)$ is relatively prime to $k(x)$ and \[f(x)k(x)=g(x)k(x)\mod p(x),\] then $f(x)\equiv g(x)\mod p(x)$.
\begin{proof}
	\dots
\end{proof}

\item[\textbf{5.1.9.}] Prove that $f(x)\equiv g(x)\mod p(x)$ if and only if $f(x)$ and $g(x)$ leave the same remainder when divided by $p(x)$.
\begin{proof}
	\dots
\end{proof}

\item[\textbf{5.1.12.}] If $f(x)$ is relatively prime to $p(x)$, prove that there exists a $g(x)\in F[x]$ such that $f(x)g(x)\equiv 1\mod p(x)$.
\begin{proof}
	\dots
\end{proof}

\end{itemize}

\end{document}
