\documentclass[12pt]{article}
\usepackage{amssymb,amsmath,amsthm}
\usepackage{graphicx,tikz}
\usepackage{fancyhdr,enumerate}
\usepackage{faktor}
\usepackage{color,soul}
\usepackage{marginnote}

\setlength{\headheight}{28pt}
\usepackage[top=1.0in, bottom=0.5in, left=0.75in, right=2.5in]{geometry}
\pagestyle{fancy}
\fancyhf{}
\lhead{MATH3004 HW \#8\\ Fall 2017}
\rhead{\thepage}
\chead{Rob Ireton}

\title{Homework 8}

\newcommand{\zee}{\mathbb{Z}}
\newcommand{\Q}{\mathbb{Q}}
\newcommand{\arr}{\mathbb{R}}
\newcommand{\C}{\mathbb{C}}
\newcommand{\such}{\text{ s.t. }}
\newcommand{\WTS}[1]{ \textcolor[rgb]{0,0.33,0}{$\blacktriangleright$ We want to show:  #1 $\blacktriangleleft$}  }

\begin{document}
\begin{itemize}
	\item[\textbf{ 1.2.34a.}] Prove that $\forall a,b,\in \zee$, $\gcd(a,b) | \gcd(a+b,a-b)$.
	\begin{proof}
	\dots
	\end{proof}

	\item[\textbf{ 4.1.2.}] Consider this subset of $\arr$:
	\[
		\zee[\pi] = \{ a_0 + a_1 \pi + \dotsb + a_n \pi^{n} \text{; } n\in\zee^{+} \text{, } {a_i}\in \zee \}\text{.}
	\]
	Show that $\zee[\pi]$ is a subring of $\arr$ that contains both $\zee$ and $\pi$.
	\begin{proof}
	\dots
	\end{proof}

	\item[\textbf{5.2.9.}] Show that $ \faktor{\arr[x]}{(x^2+1)}$ is a field by showing that every nonzero congruence class is a unit.
	[Hint: The inverse of $[ax+b]$ is $[cx+d]$ where $c=-\frac{a}{a^2+b^2}$ and $d=\frac{b}{a^2+b^2}$. Prove it.]
	\begin{proof}
	\dots
	\end{proof}

	\item[\textbf{ 5.2.13.}] Prove the first part of Theorem 5.7:
	Let $F$ be a field and $p(x)$ a nonconstant polynomial in $F[x]$. Then the set $\faktor{F[x]}{(p(x))}$ of congruence classes modulo $p(x)$ is a commutative ring with identity.
	\begin{proof}
	\dots
	\end{proof}

	\item[\textbf{ 5.3.6.}] Let $p(x)$ be irreducible in $F[x]$. If $[f(x)]\ne [0] \in \faktor{F[x]}{(p(x))}$ and $h(x)\in F[x]$, prove that there exists $g(x)\in F[x]$ such that\\ $[f(x)][g(x)]=[h(x)]$ in $\faktor{F[x]}{(p(x))}$.
	\begin{proof}
	\dots
	\end{proof}

	\item[\textbf{ 5.3.8.}] If $p(x)$ is an irreducible quadratic in $F[x]$, show that $\faktor{F[x]}{(p(x))}$ contains all the roots of $p(x)$.
	\begin{proof}
	\dots
	\end{proof}

	\item[\textbf{ 6.1.8.}] Let $R, S$ be rings. If $I$ is an ideal in $R$ and $J$ is an ideal in $S$, prove that $I\times J$ is an ideal in $R\times S$.
	\begin{proof}
	Let $i_1, i_2 \in I$ and $j_1, j_2 \in J$.
	Then $(i_1, j_1), (i_2, j_2) \in I\times J$.
	\begin{enumerate}
		\item[\textbullet] $(i_1, j_1) - (i_2, j_2) = (i_1 - i_2, j_1 - j_2)$. $I$ is an ideal in $R$, so $i_1 - i_2 \in I$. $J$ is an ideal in $S$, so $j_1 - j_2 \in J$. Therefore, $(i_1 - i_2, j_1 - j_2) \in I\times J$.
		\item[\textbullet] $(i_1, j_1) (i_2, j_2) = (i_1 i_2, j_1 j_2)$. $I$ is an ideal in $R$, so $i_1 i_2 \in I$. $J$ is an ideal in $S$, so $j_1 j_2 \in J$. Therefore, $(i_1 i_2, j_1 j_2) \in I\times J$. By a similar argument, $(i_2, j_2) (i_1, j_1)$ also in $I\times J$.
	\end{enumerate}
	Since $I\times J$ is closed under subtraction and multiplication, Theorem 6.1 (The ``Ideal Test'') lets us conclude that $I\times J$ is an ideal in $R\times S$.
	\end{proof}

	\item[\textbf{ 6.1.10.}] Let $F$ be a field. If $I$ in an ideal in $F$, prove that $I=(0)$ or $I=F$.
	\begin{proof}
	\dots
	\end{proof}

	\item[\textbf{ 6.1.15.}] Show that the ideal generated by $x$ and 2 in $\zee[x]$ is the ideal $I$ of all polynomials with even constant terms.
	\begin{proof}
	\dots
	\end{proof}

	\item[\textbf{ 6.1.17a.}] Let $R$ be a ring. If $I$ and $J$ are ideals in $R$, prove that $I\cap J$ is an ideal of $R$.
	\begin{proof}
		Let $a, b\in I\cap J$. Then $a, b \in I$ and $a, b \in J$.
		\begin{enumerate}
			\item[\textbullet] $I$ is an ideal in $R$, so $a-b \in I$. $J$ is an ideal in $R$, so $a-b \in J$. Therefore, $a-b \in I\cap J$.
			\item[\textbullet] $I$ is an ideal in $R$, so $ab \in I$. $J$ is an ideal in $R$, so $ab \in J$. Therefore, $ab \in I\cap J$. By a similar argument, $ba \in I\cap J$.
		\end{enumerate}
		By the Ideal Test, $I\cap J$ is an ideal of $R$.
	\end{proof}

	\item[\textbf{ 6.1.28.}] Let $R$ be a ring. If $I$ is an ideal of $R$, prove that $I[x]$ is an ideal in $R[x]$.
	\begin{proof}
	\dots
	\end{proof}


\end{itemize}
\end{document}
